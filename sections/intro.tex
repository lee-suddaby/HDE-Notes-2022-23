\section*{Introduction}
\addcontentsline{toc}{section}{Introduction}

Throughout history, humanity has sought the answers to deep, philosophical questions such as What is the meaning of life? What is consciousness? Do we even have free will? Why don't they like me back? Unable to answer any of these, in these notes we turn our attention to the study of ordinary and partial differential equations.

These notes are based on lectures given for the course Honours Differential Equations in academic year 2021-22. Any mistakes, typos, and omissions are invariably my own. Please report any mistakes you find to \texttt{\href{mailto:L.J.Suddaby@sms.ed.ac.uk}{L.J.Suddaby@sms.ed.ac.uk}}.

\subsection*{Acknowledgements}

Thanks to Alicia and Tom for spotting typos and their other valued contributions.

\subsection*{Notation}

$y'$ and $\dot{y}$ will both be used to denote differentiation, with the latter being used in the particular case of differentiation with respect to $t$.

$x_i(t)$, or simply $x_i$, will be used to denote the $i$th variable in a system. $\xt$ will denote the vector of the $x_i$'s, such that $\xt = \mat{x_1 \\ \vdots \\ x_n}$, and $\xtp$ is the derivative applied to $\xt$: $\xtp = \mat{x_1' \\ \vdots \\ x_n'}$. Where solutions of a system are considered, $\vbx^{(j)}(t) = \mat{x_1^{(j)} \\ \vdots \\ x_n^{(j)}}$ denotes the vector containing the $j$th solution of the system.

The first partial derivative of the function $u$ with respect to $x$ will be denoted $\p_x u$, $u_x$, or $\frac{\p u}{\p x}$, with the three being equivalent. Similarly, the second partial derivative is $\p_{xx}u = u_{xx} = \frac{\p^2 u}{\p x^2}$.